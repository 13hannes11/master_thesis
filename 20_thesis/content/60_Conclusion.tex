\chapter{Conclusion and Future Work}
\label{ch:Conclusion}

This section briefly summarizes the whole thesis, gives a conclusion. This is followed by limitations and future work.

\todo[inline, caption={}]{
    \begin{itemize}
        \item 1 - 3 pages
        \item Summarize the main results in a general level.
        \item Tell what was your own contribution and what was based on other sources.
        \item Possibly also critics (e.g., limitations), alternative approaches, topics for      future research.
        \item No more new results and seldom any references (at most for alternative, unmentioned approaches)
    \end{itemize}
}

\todo[inline]{Revise conclusion bullet points based on discussion.}

\section{Conclusion}
\label{sec:Conclusion:Conclusion}

\begin{itemize}
    \item Summary/Conclusion:
    \begin{itemize}
        \item Thesis developed a focus annotation tool to efficiently generate annotations
        \item Investigated feasibility of predicting distance to focal plane

        \item Developed CNN models that work well on the task and outperform all traditional methods
        \item However, hardware requirements are much higher for best performing models, the methods are much slower and require more memory
        \item Smaller cnn models seem to work fairly well (relate to focusliteNN) on the task but don't beat best traditional metric
        \item Fully connected models have trouble with the task
        \item Larger CNN models fare quite good at the task
    \end{itemize}
\end{itemize}

\section{Limitations and Future Work}
\label{sec:Conclusion:FutureWork}

This section first discusses limitations and how they inform future work.

\begin{itemize}
    \item Limitations
    \begin{itemize}
        \item Limited data
        \item Only one person annotated data
        \item Only one image is marked as in focus even when multiple could be
        \item No background usage, might cause models to miss important details, make stitched images to be perceived as not in focus
        \item Not running seperate optimizer runs for ResNet might make individual comparison between resnet models inaccurate
        \item Speed comparison not ideal as neither the nn nor the traditional methods were particularly optimised for speed
    \end{itemize}
\end{itemize}

\begin{itemize}
    \item Future Work:
    \begin{itemize}
        \item How would more modern models fare on the task? (ResNext, VisionTransformer, etc.)
        \item Investigate how performance can be improved (especially small models).
        \item Investigate input size agnostic models on whole images (ResNet)
        \item Investigate training models that generate heatmap for whole image based on automatically labelled data with patch based models
        \item Introduce background annotations and investigate effect on performance
        \item Investigate what features allow models to predict direction
        \item Can performance improve with more data?
    \end{itemize}
\end{itemize}
