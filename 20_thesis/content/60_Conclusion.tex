\chapter{Conclusion and Future Work}
\label{ch:Conclusion}

This section briefly summarizes the whole thesis, gives a conclusion. This is followed by limitations and future work.

\todo[inline, caption={}]{
    \begin{itemize}
        \item 1 - 3 pages
        \item Summarize the main results in a general level.
        \item Tell what was your own contribution and what was based on other sources.
        \item Possibly also critics (e.g., limitations), alternative approaches, topics for      future research.
        \item No more new results and seldom any references (at most for alternative, unmentioned approaches)
    \end{itemize}
}

\todo[inline]{Revise conclusion bullet points based on discussion.}

\section{Conclusion}
\label{sec:Conclusion:Conclusion}

In summary the goal of this thesis was to evaluate the feasibility of detecting the focus distance of microscopy slides. The dataset used are Kato-Katz prepared image slides which contain \aclp{sth} and Schistosoma eggs. As they were missing focus annotations an efficient annotation tool for annotating image patches was developed. Further, \ac{poop} models are introduced which were trained on the annotated data. These models are not only evaluated on focus distance detection but also compared to traditional metrics for selecting the most in-focus image in a focus stack. Models that perform well on focus distance detection also perform well on the other task. Also, this study shows that efficient fast models are able to learn the given task as well, however, trading accuracy for computational efficiency.
Overall, the results of this thesis show the feasibility of using machine learning models for focus distance estimation.


\section{Limitations and Future Work}
\label{sec:Conclusion:FutureWork}

This section first discusses limitations and how they inform future work.

\begin{itemize}
    \item Limitations
    \begin{itemize}
        \item Limited data
        \item Only one person annotated data
        \item Only one image is marked as in focus even when multiple could be
        \item No background usage, might cause models to miss important details, make stitched images to be perceived as not in focus
        \item Not running seperate optimizer runs for ResNet might make individual comparison between resnet models inaccurate
        \item Speed comparison not ideal as neither the nn nor the traditional methods were particularly optimised for speed
    \end{itemize}
\end{itemize}

\begin{itemize}
    \item Future Work:
    \begin{itemize}
        \item How would more modern models fare on the task? (ResNext, VisionTransformer, etc.)
        \item Investigate how performance can be improved (especially small models).
        \item Investigate input size agnostic models on whole images (ResNet)
        \item Investigate training models that generate heatmap for whole image based on automatically labelled data with patch based models
        \item Introduce background annotations and investigate effect on performance
        \item Investigate what features allow models to predict direction
        \item Can performance improve with more data?
    \end{itemize}
\end{itemize}
