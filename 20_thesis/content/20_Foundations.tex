\chapter{Foundations}
\label{ch:Foundations}

\section{Neglected Tropical Diseases}
\label{sec:Foundations:NTDs}

\ac{NTDs} \cite{feasey2010neglected} are a biologically diverse group of diseases which mainly occur in tropical regions. The geographic location, however, is not linked to temperatures or climate but mainly as a result of a large part of the world's poorest population living in these regions. \ac{NTDs} affect approximately a sixth of the world population, yet, they have received less attention than other diseases. \textcite{feasey2010neglected} state that there are \ac{NTDs} which have a high prevalence and excellent potential for successful control. Among these are Ascaris, Hookworm, Whipworm, and Schistosoma. These parasites are the ones present in the dataset used in this thesis. Ascaris, Hookworm and Whipworm are \ac{STHs} which is a subclass of \ac{NTDs}\todo{Source for subclass}.

\subsection{\acl{STHs}}
\label{sec:Foundations:NTDs:STHs}

Helminthic parasites which are mainly transmitted through soil are known as \acl{STHs} \cite{feasey2010neglected,jourdan2018soiltransmitted}. 

\ac{STHs} are as implied by the name 


This section gives a more detailed introduction into \ac{STHs} present in the dataset used.

\todo{there are additional STHs: Strongyloides stercoralis}

% Soil-Transmitted Helminths


\subsubsection{Roundworm (Ascaris)}
\label{sec:Foundations:NTDs:STHs:Ascaris}

\subsubsection{Hookworm (Necator Americanus and Ancylostoma Duodenale)}
\label{sec:Foundations:NTDs:STHs:Hookworm}

\subsubsection{Whipworm (Trichuris)}
\label{sec:Foundations:NTDs:STHs:Whipworm}

\subsection{Schistosoma}
\label{sec:Foundations:NTDs:STHs:Schistosoma}

Schistosoma are, unlike \ac{STHs}, not transmitted through soil.  \todo{Write section for schistosoma}

\subsection{Diagnosis}
\label{sec:Foundations:NTDs:Diagnosis}

Schistosoma and \ac{STHs} are commonly diagnosed by counting eggs present in a stool sample with a microscope. The technique used is called Kato-Katz \todo{cite where it is mentioned with that name} and was introduced by \textcite{katz1972simple} which refined the method introduced by \citeauthor{kato1954comparative} \cite{kato1954comparative,kato1960correct}. The technique involves cardboard with a hole in it, some stainless-steel mesh-cloth, a cellophane membrane and glycerol. A sample first is pressed through the stainless steel mesh-cloth and then smeared through the hole onto the slide. Then the side is covered with a cellophane membrane and treated with glycerol \cite{mbongngwese2020diagnostic}. This technique gains fairly consistent measurements, is fairly reliable, easy to perform and low cost \cite{katz1972simple} and therefore considered to be the \ac{WHO} \say{gold standard} (according to \textcite{mbongngwese2020diagnostic}).

The slides prepared using the Kato-Katz are, after a resting period, examined under a light microscope to count the number of eggs per milligram. This allows to identify the intensity of infection of a patient \cite{feasey2010neglected}. 

\ac{WSI} \cite{ghaznavi2013digital, hanna2019whole, el-gabry2014wholeslide} is a process in which slides are wholistically scanned to create a digital copy. This allows diagnostic processes to be more flexible as the collection of slides is independent of its analysis. Further, digitally archived slides do not suffer any degradation. 

A commonly used focus method for \ac{WSI} is \emph{z-stacking} \cite{el-gabry2014wholeslide} which is required for slides with larger variation in topology or which have a three-dimensional structure e.g., thick smears. A z-stack consists of multiple images taken at different focus planes. This allows a researcher to emulate the focusing behaviour of a conventional light microscope. \todo{Add example of a focus stack.} \todo{Add examples of eggs.}


