\chapter{Foundations}
\label{ch:Foundations}

\section{Neglected Tropical Diseases}
\label{sec:Foundations:NTDs}

\ac{NTDs} \cite{feasey2010neglected} are a biologically diverse group of diseases which mainly occur in tropical regions. The geographic location, however, is not linked to temperatures or climate but mainly as a result of a large part of the world's poorest population living in these regions. \ac{NTDs} affect approximately a sixth of the world population, yet, they have received less attention than other diseases. \textcite{feasey2010neglected} state that there are \ac{NTDs} which have a high prevalence and excellent potential for successful control. Among these are Ascaris, Hookworm, Whipworm, and Schistosoma. These parasites are the ones present in the dataset used in this thesis. Ascaris, Hookworm and Whipworm are \ac{STHs} which is a subclass of \ac{NTDs}\todo{Source for subclass}.

\subsection{\acl{STHs} and Schistosoma}
\label{sec:Foundations:NTDs:STHs}

This section gives a more detailed introduction into Schistosoma and \acl{STHs} present in the dataset used.

\todo{there are additional STHs: Strongyloides stercoralis}

% Soil-Transmitted Helminths
\label{sec:Foundations:NTDs:STHs}

\subsubsection{Roundworms (Ascaris)}
\label{sec:Foundations:NTDs:STHs:Ascaris}

\subsubsection{Hookworm (Necator Americanus and Ancylostoma Duodenale)}
\label{sec:Foundations:NTDs:STHs:Hookworm}

\subsubsection{Whipworm (Trichuris)}
\label{sec:Foundations:NTDs:STHs:Whipworm}

\subsubsection{Schistosoma Mansoni}
\label{sec:Foundations:NTDs:STHs:Schistosoma}

\subsection{Diagnosis}
\label{sec:Foundations:NTDs:Diagnosis}

The diseases mentioned in \autoref{sec:Foundations:NTDs:STHs} are commonly diagnosed \todo{finish how to diagnose including sources}
\todo{introduce kato katz}
