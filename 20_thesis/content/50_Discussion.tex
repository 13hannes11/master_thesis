\chapter{Discussion}
\label{ch:Discussion}

The key results in \autoref{ch:Results} show that using a neural network based model is feasible. \Ac{resnet} based models perform well on this task and manage to outperform traditional metrics in related tasks such as selecting the best in focus image from a focus stack. Further, the developed \ac{poop}-\ac{cnn} model performs on a similar performance level as other traditional metrics. Surprising, was that \ac{mdct} outperforms \ac{vol4} by such large extent. Due to the study by \textcite{mateos-perez2012comparative}, the expectation was to see similar performance levels between \ac{mdct} and \ac{vol4}.

\section{Outliers}
\label{ch:Discussion:Outliers}

The distance prediction of developed models might have been hampered slightly by outliers which are present in both scatter plots (see \autoref{fig:Results:ScatterPredictedTarget}). This presence of outliers indicates that these incorrect predictions are not only the result of difficult to predict images but also that some images might have incorrect target information. This could be caused by erroneous sensor readings when collecting the data. A further cause of error could be that the data was annotated by only a single person. Some images also contain eggs that are difficult to detect; therefore these may have caused incorrect annotations and predictions.

\section{Performance of Small and Large Models}
\label{ch:Discussion:SmallandLarge}

Also, on the task of distance prediction the small \ac{poop}-\ac{cnn} model does not capture the task as well as larger \ac{poop}-\ac{resnet} models. This might be due to the shallow model not being sufficiently complex for the data. Further, the \ac{poop}-\ac{fc} model did not do well on learning the task, it also was the worst method when selecting the best image from a focus-stack. A likely explanation is the missing spatial information that is explicitly present in convolution based models. With sufficient data a \ac{fc} might be able to learn important spacial relationships, but the limited amount of data did not allow that.

\section{Comparing to FocusLiteNN}
\label{ch:Discussion:FocusLiteNN}

Similar to the results from \textcite{wang2020focuslitenn} the best performing \ac{resnet} models were not the largest ones. This might be due to lack of sufficient data to make use of the additional size larger \ac{resnet} models have. Also like \citeauthor{wang2020focuslitenn} the results in this thesis show that a small \ac{cnn} is able to tackle the problem with good compromise between accuracy and speed. However, \citeauthor{wang2020focuslitenn} \textit{FocusLiteNN} have a smaller performance gap, when compared to \ac{resnet}, than observed in this thesis. A likely explanation is the different focus of this thesis as \citeauthor{wang2020focuslitenn} focus on producing a fast and small model as opposed to being a feasibility study. Another factor could be the difference in the problem as in this thesis the distance and direction to the focus plane are estimated instead of only the degree of to which an image is out-of-focus.

\section{Performance of Traditional Methods}
\label{ch:Discussion:PerformanceTraditional}

The strong performance of \ac{mdct} and the week performance of \ac{vol4} came as a surprise. A possible explanation could be the limited neighbourhood of pixels that \ac{vol4} considers. \Ac{mdct} is slower due to considering more neighbouring pixels, but this might enable it to use more information. The \Ac{laplacian} focus metric currently used in the microscopes is outperformed by \ac{mdct} and \ac{poop}-\ac{cnn} and, therefore, a switch to \ac{mdct} or to \ac{poop}-\ac{cnn} should be considered.

\section{Stitched Images}
\label{ch:Discussion:StitchedImages}

As shown in \autoref{sec:Results:HeatStiched} the \ac{poop} models can be used to generate distance plots over a whole image or even a whole slide. This could potentially give information on the height and density of a probe in the sample. Further, generated stitched images can be used to reduce the need to store as much data. Stitched images show that important parts of an image to identify an egg are in focus, the background, however, is not reliably in focus. This could cause problems when the model produces wrong predictions and an egg in a stitched image cannot be detected. Moreover, the generation of such images, is currently slow as many small patches need to be predicted. Therefore, the output of a patch-based model could be used to automatically annotate data. Such data then can be used to train models that give a focus-distance height map as model output.

\section{Speed Considerations for Application}
\label{ch:Discussion:Applications}

In terms of speed traditional methods outperform \ac{poop} models substantially. Therefore, \ac{poop}-\ac{resnet} models cannot be used in the same domain as traditional models. \Ac{poop}-\ac{resnet} can due to their speed limitations only be used for post-processing and not directly during scanning. \Ac{poop}-\ac{cnn} can compete with traditional methods; therefore it is worth considering for similar applications. Further, of note is that the improved performance of \Ac{poop}-\ac{resnet} comes at a high computational performance. Moreover, not all models seem to equally benefit from running on a \ac{gpu}. This is likely due to the \ac{gpu}'s parallelism not being fully utilized by smaller models.
