\begin{abstract}
    Neglected tropical diseases affect approximately one-sixth of the global population. Most, are treatable with widely available medicine, however, diagnosis is expensive and slow. An automated solution using low-cost microscopes and machine learning could alleviate this. 
    During the automatic scanning process, stacks of images, focus stacks, of slides are collected. This results in a large amount of data. Machine learning models that predict the distance to the focal plane could be used, either during scanning or in post-processing, to reduce the amount of data collected. Potentially, they could also speed up the scanning process. This thesis investigates the feasibility of using such models to detect the distance and direction of the focal plane for Kato-Katz prepared image slides. Further, the developed models are compared to traditional focus metrics on an adjacent task, selecting the most in-focus image of a focus stack. 
    The results show that convolutional neural networks are able to predict the focus distance. Smaller models that can be used during scanning, instead of post-processing, trade speed for accuracy. The developed models are a starting point for further research on reducing the data generated by whole slide microscopy.
\end{abstract}
