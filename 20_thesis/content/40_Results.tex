\chapter{Results}
\label{ch:Results}

\todo[inline, caption={}]{
    \begin{itemize}
        \item Shows the findings of the project, often in the form of data.
        \item Comments on those findings which illustrate the significance of the results.
        \item Data Commentary:
        \begin{itemize}
            \item It is important to comment on the visual information, highlighting to your reader what they should notice and why this is important. 
            \item Location statement
            \item Linking as-statement
            \item Highlighting statement: drawing attention to key information.
        \end{itemize} 
    \end{itemize}
}

\begin{figure}
    \centering
    \includegraphics[width=\textwidth]{images/40_results/scatter_predictions_targets.pdf}
    \caption{The predictions plotted compared to targets with the ideal drawn as a line.}
    \label{fig:Results:ScatterPredictedTarget}
\end{figure}
\begin{figure}
    \centering
    \includegraphics[width=\textwidth]{images/40_results/histogram_predictions_targets.pdf}
    \caption{A histogram showing the distribution of predicted focus compared to annotated focus.}
    \label{fig:Results:HistogramPredictedTarget}
\end{figure}

\begin{table}
    \centering
    \caption{Comparing \acs{poop} to other methods in terms of accuracy of finding the best in-focus image ($\pm 1$) from a focus stack.}
    \begin{tabular}{|l|rrrrr|}
        \hline
        \multicolumn{1}{|c}{} & \multicolumn{5}{|c|}{accuracy} \\
        name            & all & trichuris & ascaris & schistosoma & hookworm \\
        \hline
        \acs{mdct}      & 0.4032 & 0.3125 & 0.6000 & 0.3125 & 0.4000 \\
        \acs{vol4}      & 0.4355 & 0.5000 & 0.6000 & 0.5000 & 0.1333 \\
        \Acs{laplacian} & 0.5968 & 0.5000 & 0.7333 & 0.6250 & 0.5333 \\
        \hline
        FullyConnected  & 0.1935 & 0.2500 & 0.2000 & 0.1250 & 0.2000 \\
        Conv            & 0.7419 & 0.6875 & 0.5333 & 0.8750 & 0.8667 \\
        ResNet-101      & 0.8226 & 0.6875 & 0.8000 & 0.8750 & 0.9333 \\
        ResNet-34       & 0.9032 & 0.9375 & 0.8667 & 0.9375 & 0.8667 \\
        ResNet-18       & 0.8226 & 0.6875 & 0.7333 & 0.8750 & 1.0000 \\ 
        \hline
    \end{tabular}
    \label{tab:Results:Comparison:RelatedWorks:Accuracy}
\end{table}

\begin{table}
    \centering
    \caption{Comparing \acs{poop} to other methods in terms of \ac{mae} of image indexes compared to the best in-focus image from a focus stack.}
    \begin{tabular}{|l|rrrrr|}
        \hline
        \multicolumn{1}{|c}{} & \multicolumn{5}{|c|}{\ac{mae}} \\
        name & all & trichuris & ascaris & schistosoma & hookworm \\
        \hline
        \acs{mdct}      & 2.3065 & 2.1250 & 1.9333 & 2.6250 & 2.5333 \\
        \acs{vol4}      & 2.2258 & 1.8750 & 1.8000 & 2.1250 & 3.1333 \\
        \Acs{laplacian} & 1.5968 & 1.8125 & 1.3333 & 1.5625 & 1.6667 \\
        \hline
        FullyConnected  & 3.2742 & 3.3125 & 3.4667 & 3.0000 & 3.3333 \\
        Conv            & 1.0645 & 1.3750 & 1.4000 & 0.7500 & 0.7333 \\
        ResNet-101      & 0.9194 & 1.0625 & 1.0667 & 0.9375 & 0.6000 \\
        ResNet-34       & 0.6452 & 0.6250 & 0.8000 & 0.5625 & 0.6000 \\
        ResNet-18       & 0.9516 & 1.3125 & 0.8667 & 1.0625 & 0.5333 \\
        \hline
    \end{tabular}
    \label{tab:Results:Comparison:RelatedWorks:IndexMAE}
\end{table}



\begin{figure}
    \centering
    \begin{subfigure}[b]{\textwidth}
        \centering
        \caption{stack index: 4}
        \includegraphics[width=\textwidth]{images/40_results/focus_stack_4.jpg}
        \label{fig:Results:Stack:SideBySide:Index0}
    \end{subfigure}
    \begin{subfigure}[b]{\textwidth}
        \centering
        \caption{stack index: 5}
        \includegraphics[width=\textwidth]{images/40_results/focus_stack_5.jpg}
        \label{fig:Results:Stack:SideBySide:Index1}
    \end{subfigure}
    \begin{subfigure}[b]{\textwidth}
        \centering
        \caption{stack index: 6}
        \includegraphics[width=\textwidth]{images/40_results/focus_stack_6.jpg}
        \label{fig:Results:Stack:SideBySide:Index2}
    \end{subfigure}
    \begin{subfigure}[b]{\textwidth}
        \centering
        \caption{stitched image}
        \includegraphics[width=\textwidth]{images/40_results/focus_stack_stiched.png}
        \label{fig:Results:Stack:SideBySide:Stiched}
    \end{subfigure}
    \caption{Comparison of images from the focus stack to a generated combined image based on the focus prediction of the model.}
    \label{fig:Results:Stack:SideBySide}
\end{figure}


\begin{figure}
    \centering
    \includegraphics[width=\textwidth]{images/40_results/focus_stack_contour_plot.pdf}
    \caption{Contour plot showing which image of the focus stack is most in focus according to the model.}
    \label{fig:Results:Stack:ContourPlot}
\end{figure}

\begin{figure}
    \centering
    \begin{subfigure}[b]{\textwidth}
        \centering
        \includegraphics[width=\textwidth]{images/40_results/heatmap_legend.pdf}
    \end{subfigure}
    \begin{subfigure}[b]{0.5\textwidth}
        \centering
        \caption{stack index 3}
        \includegraphics[width=\textwidth]{images/40_results/heatmap_layer_3.pdf}
        \label{fig:Results:Stack:HeatMap:Stack3}
    \end{subfigure}%
    \begin{subfigure}[b]{0.5\textwidth}
        \centering
        \caption{stack index 4}
        \includegraphics[width=\textwidth]{images/40_results/heatmap_layer_4.pdf}
        \label{fig:Results:Stack:HeatMap:Stack4}
    \end{subfigure}
    \par
    \begin{subfigure}[b]{0.5\textwidth}
        \centering
        \caption{stack index 5}
        \includegraphics[width=\textwidth]{images/40_results/heatmap_layer_5.pdf}
        \label{fig:Results:Stack:HeatMap:Stack5}
    \end{subfigure}%
    \begin{subfigure}[b]{0.5\textwidth}
        \centering
        \caption{stitched image}
        \includegraphics[width=\textwidth]{images/40_results/heatmap_stiched.pdf}
        \label{fig:Results:Stack:HeatMap:Stiched}
    \end{subfigure}
    \caption{Figure comparing the heatmap of images of a focus stack to a generated image based on the focus prediction model.}
    \label{fig:Results:Stack:HeatMap}
\end{figure}
