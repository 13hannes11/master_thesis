\chapter{Results}
\label{ch:Results}

\todo[inline, caption={}]{
    \begin{itemize}
        \item Shows the findings of the project, often in the form of data.
        \item Comments on those findings which illustrate the significance of the results.
        \item Data Commentary:
        \begin{itemize}
            \item It is important to comment on the visual information, highlighting to your reader what they should notice and why this is important. 
            \item Location statement
            \item Linking as-statement
            \item Highlighting statement: drawing attention to key information.
        \end{itemize} 
    \end{itemize}
}

\begin{figure}
    \centering
    \includegraphics[width=\textwidth]{images/40_results/scatter_predictions_targets.pdf}
    \caption{The predictions plotted compared to targets with the ideal drawn as a line.}
    \label{fig:Results:ScatterPredictedTarget}
\end{figure}
\begin{figure}
    \centering
    \includegraphics[width=\textwidth]{images/40_results/histogram_predictions_targets.pdf}
    \caption{A histogram showing the distribution of predicted focus compared to annotated focus.}
    \label{fig:Results:HistogramPredictedTarget}
\end{figure}


\begin{figure}
    \centering
    \begin{subfigure}[b]{\textwidth}
        \centering
        \caption{stack index: 4}
        \includegraphics[width=\textwidth]{images/40_results/focus_stack_4.jpg}
        \label{fig:Results:Stack:SideBySide:Index0}
    \end{subfigure}
    \begin{subfigure}[b]{\textwidth}
        \centering
        \caption{stack index: 5}
        \includegraphics[width=\textwidth]{images/40_results/focus_stack_5.jpg}
        \label{fig:Results:Stack:SideBySide:Index1}
    \end{subfigure}
    \begin{subfigure}[b]{\textwidth}
        \centering
        \caption{stack index: 6}
        \includegraphics[width=\textwidth]{images/40_results/focus_stack_6.jpg}
        \label{fig:Results:Stack:SideBySide:Index2}
    \end{subfigure}
    \begin{subfigure}[b]{\textwidth}
        \centering
        \caption{Combined image based on heatmap.}
        \includegraphics[width=\textwidth]{images/40_results/focus_stack_stiched.png}
        \label{fig:Results:Stack:SideBySide:Stiched}
    \end{subfigure}
    \caption{Comparison of images from the focus stack to a generated combined image based on the focus prediction of the model.}
    \label{fig:Results:Stack:SideBySide}
\end{figure}


\begin{figure}
    \centering
    \includegraphics[width=\textwidth]{images/40_results/focus_stack_contour_plot.pdf}
    \caption{Contour plot showing which image of the focus stack is most in focus according to the model.}
    \label{fig:Results:Stack:ContourPlot}
\end{figure}

\begin{figure}
    \centering
    \begin{subfigure}[b]{\textwidth}
        \centering
        \includegraphics[width=\textwidth]{images/40_results/heatmap_legend.pdf}
    \end{subfigure}
    \begin{subfigure}[b]{0.5\textwidth}
        \centering
        \caption{stack index 3}
        \includegraphics[width=\textwidth]{images/40_results/heatmap_layer_3.pdf}
        \label{fig:Results:Stack:HeatMap:Stack3}
    \end{subfigure}%
    \begin{subfigure}[b]{0.5\textwidth}
        \centering
        \caption{stack index 4}
        \includegraphics[width=\textwidth]{images/40_results/heatmap_layer_4.pdf}
        \label{fig:Results:Stack:HeatMap:Stack4}
    \end{subfigure}
    \par
    \begin{subfigure}[b]{0.5\textwidth}
        \centering
        \caption{stack index 5}
        \includegraphics[width=\textwidth]{images/40_results/heatmap_layer_5.pdf}
        \label{fig:Results:Stack:HeatMap:Stack5}
    \end{subfigure}%
    \begin{subfigure}[b]{0.5\textwidth}
        \centering
        \caption{stitched image}
        \includegraphics[width=\textwidth]{images/40_results/heatmap_stiched.pdf}
        \label{fig:Results:Stack:HeatMap:Stiched}
    \end{subfigure}
    \caption{Figure comparing the heatmap of images of a focus stack to a generated image based on the focus prediction model.}
    \label{fig:Results:Stack:HeatMap}
\end{figure}
