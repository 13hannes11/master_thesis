\chapter{Introduction}
\label{ch:Introduction}

Approximately one-sixth of the world population is affected by a set of diseases classified as \acfp{ntd} \cite{feasey2010neglected}. The most prevalent of those are \acfp{sth} and schistosomiasis \cite{barenbold2017estimating}. Numerous \acp{ntd} are preventable with proper access to safe water, sanitation, and appropriate housing. Moreover, mass drug administration treatments have been conducted to reduce the severity and prevalence of infections in vulnerable populations \cite{jourdan2018soiltransmitted}. The monitoring of prevalence of diseases and the elimination using mass drug administration requires highly sensitive methods that often are not available in poor countries which is why cost-effective but sensitive methods are required \cite{mbongngwese2020diagnostic}.
Traditional methods count eggs under a microscope, this, however, this is expensive on larger scales. Moreover, it suffers from setting-dependent sensitivity and decay of eggs in the sample \cite{barenbold2017estimating}. 

In a joined consortium \todo{maybe cite project hete \url{https://ai4ntd.org/}}, Etteplan Sweden \cite{etteplanoyjengineering} is developing low-cost microscopes to be used in the field on site. The aim is to develop a machine learning based approach that reaches a sensitivity and specificity of 95 percent 
\cite{etteplan2021fighting}. Using low-cost automatic microscopes for in-field scanning aids with high-staff turnover rates, reduces training time, provides an easily scalable solution and eliminates transport related egg degeneration.
The scanning process produces a large amount of image data as each location on a microscopy slide is scanned at multiple focus distances.

This thesis proposes a machine learning based approach (see \autoref{ch:Methods}), \ac{poop}, to predict the distance to the optimal focus plane of a microscopy image patch. A data driven, approach requires annotated data. Thus, in this thesis, first, suitable data is extracted from a large database of images. An annotation tool is developed to facilitate efficient annotations of focus-data. Further, machine learning models are developed based on the annotated data. The developed models are evaluated and compared to traditional focus metrics used in microscopy, which is presented in \autoref{ch:Results}.

The purpose of this research is show the feasibility of using focus prediction models on faecal smear samples instead of tissues samples as done by related approaches, introduced in \autoref{ch:Foundations}. Faecal smear samples face different challenges as they include debris not present in tissue samples.
A further objective is to also show that not only the degree to which an image is out of focus but also the direction and distance to the focal plane can be determined. The development of versatile models, allows the use in applications that are reliant on speed but also for applications with a high requirement for accuracy. As an example \ac{poop} models could find use while scanning to improve scan times. They could be used during or post scanning to reduce collected data. Post-processing also allows the creation of in-focus mosaics images which can aid in egg detection and speed up sample inspection.
