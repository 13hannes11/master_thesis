\documentclass[a4paper,11pt]{article}
\usepackage[T1]{fontenc}
\usepackage[utf8]{inputenc}
\usepackage{lmodern}
\usepackage{amsmath}
\usepackage{amsfonts}
\usepackage{amssymb}
\usepackage{amsthm}
\usepackage{graphicx}
\usepackage{url}
\usepackage{listings}
\usepackage{hyperref}
\usepackage{parskip}

\usepackage[citestyle=numeric,style=numeric,sorting=none, maxnames=2, giveninits=true, backend=biber]{biblatex}
\addbibresource{bibliography.bib}

\title{Automatic Assessment of Focus Quality in Microscopy Images}
\author{Hannes F. Kuchelmeister}
\date{\today}

\begin{document}

\maketitle
\newpage
%\tableofcontents
%\newpage

%\begin{abstract}
%\end{abstract}


\section{Background}

Here you describe in what context your thesis is to be done. What prerequisites are valid, what is the goal of the project from the supervisors point of view, what is available and has been done before, under what circumstances should the work be done.

\section{Description of the task}

Here you describe in more detail the contents of the project: what should be done and what moments are included. Specifically it should be described the interesting part of the problem, how this should be analyzed and solved. Here there should be clarified that the project fulfills the requirements that the university has posed on a thesis project on this level.

\section{Methods}

What systems, tools and methods should be used. Relevant literature (it is often a part of the work to find additional literature). How results should be evaluated and documented.

\section{Relevant courses}

A list of courses that are of particularly interest for the project.

\section{Delimitations}

It is also important to specify what is not part of the project. This prevents that the thesis grows uncontrolled. You can add stuff that you can do if there are time left, also write down things that can be skipped if there are not time for them.

\section{Time plan}

Here you find the time plan for the project. When to start and how much time has been allocated for each moment. (the fine grading can vary, but no blocks bigger that about 4 weeks should be listed). The time plan should take into consideration factors such as part time work, vacations, other obligations etc. Some moments, like writing, is usually done in parallel with other tasks. A graphical plan is encouraged. Regular meetings with your reviewer should be booked.



\printbibliography[heading=bibintoc]
\end{document}