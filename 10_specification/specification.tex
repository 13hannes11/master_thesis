\documentclass[a4paper,11pt]{article}
\usepackage[T1]{fontenc}
\usepackage[utf8]{inputenc}
\usepackage{lmodern}
\usepackage{amsmath}
\usepackage{amsfonts}
\usepackage{amssymb}
\usepackage{amsthm}
\usepackage{graphicx}
\usepackage{url}
\usepackage{listings}
\usepackage{hyperref}
\usepackage{parskip}
\usepackage{todonotes}
\usepackage{caption}
\usepackage{subcaption}
\usepackage{float}

\usepackage[citestyle=numeric,style=numeric,sorting=none, maxnames=2, giveninits=true, backend=biber]{biblatex}
\addbibresource{bibliography.bib}

\title{Automatic Assessment of Focus Quality in Microscopy Images}
\author{Hannes F. Kuchelmeister}
\date{\today}

\begin{document}

\maketitle
\newpage
%\tableofcontents
%\newpage

%\begin{abstract}
%\end{abstract}


\section{Background}

Neglected Tropical Diseases (NTDs) are a set of diseases mainly present in tropical and subtropical regions of the world, affecting more than a billion people across the globe. Almost exclusively affecting poor regions, many of the diseases can be prevented or even completely eradicated given proper access to existing technologies and tools. However, affected populations are often of low socio-economic status and of low priority in public health efforts. One particular class of NTDs are Soil-Transmitted Helminths (STHs): Ascaris, whipworm and hookworm. These are intestinal, parasitic worms transmitted through contaminated soil. Each type of STH is estimated to affect between 500 million to 1 billion people on a global scale. As intestinal worms, the eggs of STHs are spread through the faeces of the infected. 

Part of the effort to detect STH infections has been to look for parasite eggs under a microscope. Specifically, stool samples are examined, where egg counts determine infection status. However, due to long training time of staff and high turnover rate, an automated solution could prove to be much more efficient. Such a solution is under development at Etteplan Sweden.

\todo[inline]{Here you describe in what context your thesis is to be done. What prerequisites are valid, what is the goal of the project from the supervisors point of view, what is available and has been done before, under what circumstances should the work be done.}

\section{Description of the task}

\todo[inline]{change description of the task to better fit.}

Etteplan’s solution, an automated microscope, scans glass slides with samples prepared according to the Kato-Katz technique \cite{worldhealthorganization1991basic}, in a whole slide imaging fashion \cite{hanna2019whole}. A slide is partitioned into a grid, where each grid cell corresponds to the camera’s field of view in that position. The sample has the obvious spatial extension tangential to the glass plane, but also a non-neglectable spatial distribution along the optical axis, compared to the cameras depth of field. Hence, the camera takes multiple images of the grid cell, varying the focal plane, to capture as much of the information of the sample in each grid cell, a process called z-stacking. The result is multiple images of each grid cell, with different regions of the image being in focus for different positions of the focal plane.

A sample can be seen in \autoref{fig:focusComparison}. This z-stacking procedure has the drawback that it produces much more data and time spent scanning each grid cell, and hence an automatic procedure to determine the degree of out-of-focus of regions of a given image could be beneficial to reduce these cost factors. Also, determining the distance and direction to the focal plane for a given out-of-focus region could be beneficial. A possible statement of the problem could be: let $M_{m,n}(R)$ be the set of real-valued m-by-n matrices (image with height m and width n). Construct a function $f : M_{m,n}(R)\rightarrow M_{m,n}(R)$ such that the elements in the image of $f$ are some metric to the optimal focal plane for that element.

\begin{figure}
    \begin{subfigure}[t]{0.3\textwidth}
        \centering
        \includegraphics[width=\textwidth]{./img/out_of_focus_hither.png}
        \caption{Out-of-focus image with focal plane hither to sample.}
        \label{fig:focusHither}
    \end{subfigure}
    \hfill
    \begin{subfigure}[t]{0.3\textwidth}
        \centering
        \includegraphics[width=\textwidth]{./img/subregion_focus.png}
        \caption{Image with subregions in focus.}
        \label{fig:focus}
    \end{subfigure}
    \hfill
    \begin{subfigure}[t]{0.3\textwidth}
        \centering
        \includegraphics[width=\textwidth]{./img/out_of_focus_thither.png}
        \caption{Out-of-focus image with focal plane thither to sample.}
        \label{fig:focusThither}
    \end{subfigure}
    \caption{Three image samples.}
    \label{fig:focusComparison}
\end{figure}


\todo[inline]{Here you describe in more detail the contents of the project: what should be done and what moments are included. Specifically it should be described the interesting part of the problem, how this should be analyzed and solved. Here there should be clarified that the project fulfills the requirements that the university has posed on a thesis project on this level.}

\section{Methods}

\todo[inline]{What systems, tools and methods should be used. Relevant literature (it is often a part of the work to find additional literature). How results should be evaluated and documented.}

\subsection{Training Data}

\subsection{Literature}


\section{Relevant courses}

\begin{table}[H]
    \centering
    \begin{tabular}{|c|c|r|} 
        \hline
        Code & Name & ECTS \\
        \hline
        \texttt{1MD120} & Deep Learning for Image Analysis & 7.5 \\
        \texttt{1DL073} & Natural Computation Methods for Machine Learning & 10.0 \\
        \texttt{1TD396} & Computer-Assisted Image Analysis I & 5.0\\
        \texttt{1DL340} & Artificial Intelligence & 5.0\\
        \hline
    \end{tabular}
\end{table}

\section{Delimitations}

\todo[inline]{It is also important to specify what is not part of the project. This prevents that the thesis grows uncontrolled. You can add stuff that you can do if there are time left, also write down things that can be skipped if there are not time for them.}

\section{Time plan}

\todo[inline]{Here you find the time plan for the project. When to start and how much time has been allocated for each moment. (the fine grading can vary, but no blocks bigger that about 4 weeks should be listed). The time plan should take into consideration factors such as part time work, vacations, other obligations etc. Some moments, like writing, is usually done in parallel with other tasks. A graphical plan is encouraged. Regular meetings with your reviewer should be booked.}




\printbibliography[heading=bibintoc]
\end{document}